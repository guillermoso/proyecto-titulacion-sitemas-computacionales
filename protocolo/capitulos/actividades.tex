\section{Programa de Actividades}

[Describe las actividades a realizar y señala los meses necesarios para el logro de las metas establecidas.
Es un listado que muestra la secuencia de acciones, relacionadas con la metodología, que se prevé llevar a cabo, la duración que tendrán y las fechas en que ocurrirán, demostrando siempre que terminarás en tiempo y forma.]


%Agregar dos oraciones a los parrafos
[De acuerdo a la metodología planteada, se organizan las actividades en base a un periodo de dos semestres correspondientes a la duración del proyecto, cada tarea particular tiene un tiempo aproximado que permita llegar al objetivo de la investigación.
A continuación se muestra un cronograma de actividades como ejemplo. 
Además, vale la pena mencionar que existen varios portales en la web donde puedes construir tu tabla \LaTeX de una manera visual, y luego generas y copias el código fuente de la tabla terminada y con estilo a tu documento. Uno de dichos portales es \url{http://www.tablesgenerator.com}]

\begin{table}
\centering
\resizebox*{!}{10 cm}{
\begin{tabular}{|p{4cm}|c|c|c|c|c|c|c|c|c|c|c|c|c|c|c|c|c|c|c|c|c|c|c|c}
	\hline 
	&\rotatebox{90}{Agosto 2016}
	&\rotatebox{90}{Septiembre 2016}
	&\rotatebox{90}{Octubre 2016} &\rotatebox{90}{Noviembre 2016} &\rotatebox{90}{Diciembre 2016} &\rotatebox{90}{Enero 2017} &\rotatebox{90}{Febrero 2017} &\rotatebox{90}{Marzo 2017} &\rotatebox{90}{Abril 2017} &\rotatebox{90}{Mayo 2017} &\rotatebox{90}{Junio 2017}
	&\rotatebox{90}{Julio 2017}
	&\rotatebox{90}{Agosto 2017}
	&\rotatebox{90}{Septiembre 2017}
	&\rotatebox{90}{Octubre 2017}
	&\rotatebox{90}{Noviembre 2017}
	&\rotatebox{90}{Diciembre 2017}
	&\rotatebox{90}{Enero 2018}
	&\rotatebox{90}{Febrero 2018}
	&\rotatebox{90}{Marzo 2018}
	&\rotatebox{90}{Abril 2018}
	&\rotatebox{90}{Mayo 2018}
	&\rotatebox{90}{Junio 2018}\\
	\hline
	
	Revisión del estado de la técnica&\checkmark 
	&\checkmark  &\checkmark  &\checkmark  &\checkmark  &\checkmark  &  &  &  &  &  &  &  &  &  &  &  &  &  &  &  &  &\\  
	\hline 
	Selección de la plataforma donde funcionará la aplicación&  &  &  &\checkmark  &\checkmark  &\checkmark  &  &  &  &  &  &  &  &  &  &  &  &  &  &  &  &  &\\ 
	\hline 
	Investigación y aprendizaje de ASP, autómatas a pila y gramáticas&  &  &\checkmark  &\checkmark  &\checkmark  &\checkmark  &\checkmark  &\checkmark  &\checkmark  &\checkmark  &\checkmark  &\checkmark  &  &  &  &  &  &  &  &  &  &  &\\ 
	\hline 
	Documentación de propuesta&  &  &  &  &\checkmark  &\checkmark  &\checkmark  &\checkmark  &\checkmark  &\checkmark  &\checkmark  &\checkmark  &\checkmark  &\checkmark  &\checkmark  &\checkmark  &\checkmark  &\checkmark  &\checkmark  &\checkmark  &\checkmark  &\checkmark  &\checkmark\\ 
	\hline 
	Búsqueda de opciones para reconocimiento de voz&  &  &  &  &\checkmark  &\checkmark  &\checkmark  &  &  &  &  &  &  &  &  &  &  &  &  &  &  &  &\\ 
	\hline 
	Determinar la herramienta para reconocimiento de voz&  &  &  &  &\checkmark  &\checkmark  &  &  &  &  &  &  &  &  &  &  &  &  &  &  &  &  &\\ 
	\hline 
	Integración del modulo de reconocimiento de voz&  &  &  &  &  &\checkmark  &\checkmark  &  &  &  &  &  &  &  &  &  &  &  &  &  &  &  &\\ 
	\hline 
	Pruebas del módulo de reconocimiento de voz&  &  &  &  &  &  &  &\checkmark  &  &  &  &  &  &  &  &  &  &  &  &  &  &  &\\ 
	\hline  
	Evaluar Freeling y OpenNLP como herramientas para el procesado del lenguaje natural&  &  &  &  &  &  &  &  &\checkmark  &  &  &  &  &  &  &  &  &  &  &  &  &  &\\ 
	 \hline 
	Análisis módulo de procesado de lenguaje natural&  &  &  &  &  &  &  &  &   &\checkmark  &\checkmark  &\checkmark  &  &  &  &  &  &  &  &  &  &  &\\ 
	\hline
	Diseño de módulo de procesado de lenguaje natural&  &  &  &  &  &  &  &  &   &  &  &  & \checkmark &\checkmark  &\checkmark  &  &  &  &  &  &  &  &\\ 
	\hline  
	Codificación de módulo de procesado de lenguaje natural&  &  &  &  &  &  &  &  &   &  &  &  &  &  &  &\checkmark  &\checkmark  &\checkmark  &  &  &  &  &\\ 
	\hline  
	Pruebas de módulo de procesado de lenguaje natural&  &  &  &  &  &  &  &  &   &  &  &  &  &  &  &  &  &  &\checkmark  &  &  &  &\\ 
	\hline  
	Análisis, diseño, codificación y pruebas de las interfaces visuales&  &  &  &  &  &  &  &  &   &  &  &  &  &  &  &  &  &  &  &\checkmark  &\checkmark  &  &\\ 
	\hline 
	Integración de los módulos para del prototipo.&  &  &  &  &  &  &  &  &   &  &  &  &  &  &  &  &  &  &  &  &  &\checkmark  &\\ 
	\hline
	Pruebas de integración y pruebas funcionales del prototipo.&  &  &  &  &  &  &  &  &   &  &  &  &  &  &  &  &  &  &  &  &  &  &\checkmark\\ 
	\hline
	
\end{tabular}
}
    \caption{Actividades a 24 meses}
    \label{xx}
\end{table}


[A continuación mostramos otro ejemplo:]

\begin{table}[h]
    \centering
\resizebox*{!}{8 cm}{
\begin{tabular}{|p{6cm}|c|c|c|c|c|c|c|c|c|}
	\hline 
	ACTIVIDAD&\rotatebox{90}{Febrero 2018}
	&\rotatebox{90}{Marzo}
	&\rotatebox{90}{Abril} 
    &\rotatebox{90}{Mayo} 
    &\rotatebox{90}{VACACIONES} 
    &\rotatebox{90}{Agosto} 
    &\rotatebox{90}{Septiembre} 
    &\rotatebox{90}{Octubre} 
    &\rotatebox{90}{Noviembre 2018}\\
	\hline
	Revisión de la Literatura&\checkmark 
	&\checkmark  &\checkmark  &  &  &  &  & &  \\  
	\hline
	Protocolo&\checkmark 
	&\checkmark  &\checkmark  &  &  &  &  & &  \\  
	\hline 
	Selección de software existente&  &  &  &\checkmark  &\checkmark  &\checkmark  &  &  &  \\ 
	\hline 
	Investigación y aprendizaje de Fuzzy Answer Sets&  &  &\checkmark  &\checkmark  &\checkmark  &\checkmark  &\checkmark  &  & \\ 
	\hline 
	Documentación de propuesta&  &  & \checkmark & \checkmark &\checkmark  &\checkmark  &\checkmark  &\checkmark  &\checkmark  \\ 
	\hline 
	Formulación de semántica para razonamiento difuso&  &  &  &  &\checkmark  &\checkmark  &  &  &  \\ 
	\hline 
	Evaluación de software existente&  &  & \checkmark &  &  &  &  &  &  \\ 
    \hline
	Codificación de prototipo&  &  &  &  & \checkmark & \checkmark &  &  &   \\ 
	\hline  
	Pruebas de prototipo&  &  &  &  &  &  & \checkmark &  &   \\ 
	\hline 
	Presentación y defensa de trabajo&  &  &  &  &  &  &  &  & \checkmark  \\ 
	\hline
\end{tabular}}
    \caption{Actividades a nueve meses}
    \label{nueve}
\end{table}