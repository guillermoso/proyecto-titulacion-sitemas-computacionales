\section{Planteamiento del problema}

\subsection{Antecedentes}

"La agricultura es la base de la economía en muchos países, esta provee a la humanidad de algunas de sus necesidades más básicas: comida y fibra" \cite{appsremotesensing}. 
La demanda de comida y productos derivados de la agricultura está proyectada a incrementar en un 70\% para el año 2050 \cite{wik_pingali_brocai_2008}. 
La agricultura de precisión (PA por sus siglas en inglés) será clave para poder alcanzar la agricultura sustentable. La PA se puede definir como una 
estrategia de manejo que implementa la recopilación de información, comunicación entre sistemas y técnicas de análisis de datos para apoyar la toma 
de decisiones, por ejemplo: aplicación de fertilizantes, control de plagas, control de riego, identificación de enfermedades, entre otros \cite{appsremotesensing}.

Actualmente existen varios sistemas en el mercado para el control y monitoreo de invernaderos o cuartos de producción. Estos sistemas ofrecen monitoreo de temperatura, humedad, CO2, 
e intensidad de la luz, con un costo de \$2,159.90 USD y \$2,351.00 USD respectivamente \cite{intellidose_kit_2021}, \cite{smartbee_kit_2021}. Ambos sistemas se pueden conectar 
a internet para realizar un monitoreo remoto, donde \cite{intellidose_kit_2021} necesitará una interfaz y una subscripción extra, añadiendo \$279.00 USD al costo. Existen otras 
alternativas, como \cite{ceres_greenhouse_solutions_2021}, \cite{autogrow_climate_control_2021}, \cite{climate_control_2021} 
% hacer u custom quote a esas citas

La comunicación inalámbrica entre sistemas ha cambiado los estándares de comunicación a día de hoy y la agricultura no se ha quedado atrás. 
"Para incrementar la eficiencia, productividad y reducir la intervención humana, tiempo y costo existe una necesidad de prestar atención a una
nueva tecnología llamada Internet de las Cosas (\textit{IoT} por sus siglas en inglés)" \cite{agriculture_automation_review}. El IoT es la red de dispositivos 
que adquieren y actúan sobre información sin la necesidad de que un humano intervenga \cite{agriculture_automation_review}. 

Los principales componentes de un sistema basado en \textit{IoT} se pueden dividir en 4 capas: dispositivos, red, servicios y aplicación. 
El \textit{hardware} a implementar es de suma importancia debido a que impacta directamente el costo de la implementación y restringe 
las tecnologías disponibles. Entre los dispositivos más utilizados están los diferentes Arduinos, Raspberry Pis y el microcontrolador ESP. 
Los datos obtenidos a través de estos dispositivos son (en su mayoría) enviados a una ubicación central a través de una red alámbrica o inalámbrica.
Los principales protocolos de red implementados en las redes inalámbricas son: Wi-Fi, Bluetooth, Zigbee y LoRaWAN \cite{systematicreviewiot}. 

En \cite{olive_orchard_monitorization} se implementó una red inalámbrica de sensores de bajo costo para monitorear la temperatura, humedad y concentración de gases 
del ambiente, además de la humedad del suelo en una plantación de olivos. Se utilizó el sensor DHT 11 para monitorear la temperatura y humedad del ambiente, un
higrómetro para medir la humedad del suelo y un MQ-135 para detectar gases nocivos (CO2, amoniaco, benzeno, alcohol y humo del fuego). Cada nodo de esta 
red cuenta con los sensores para realizar las lecturas mencionadas, un microcontrolador y una antena para habilitar la comunicación mediante el protocolo 
LoRaWAN, todo alimentado por celdas fotovoltaicas. Para el monitoreo de las lecturas, se desarrollaron una aplicación móvil y una interfaz web que 
muestran los datos en tiempo real.

Es importante monitorear de forma separada la humedad del ambiente como la humedad del suelo debido a que estas no están correlacionadas. Ambas se relacionan
a enfermedades y plagas, sin embargo, la humedad del ambiente influye en la aparición de enfermedades y la humedad del suelo en la aparición de plagas. También
es muy valioso contar con mediciones de temperatura cerca del cultivo. De esta manera se tienen datos más precisos que los que proveen los sitios/aplicaciones 
meteorológicas \cite{olive_orchard_monitorization}.

En la Universidad de Magdalena, Colombia se creó un sistema de bajo costo para monitoreo y control remoto de un invernadero utilizando lógica difusa 
\cite{low_cost_fuzzy_logic_greenhouse}. La utilización de lógica difusa en sistemas de control permite traducir variables a sets previamente definidos que contienen la 
terminología difusa como muy frío, frío, caliente, muy caliente, etcétera. Gracias a esta traducción se pueden tomar acciones más precisas para realizar el 
control \cite{agriculture_automation_review}. El algoritmo difuso se implementó en un Arduino Mega para tomar acciones de control sobre la temperatura y humedad 
del ambiente, humedad del suelo e iluminación. Además, se diseño un sitio web para realizar el monitoreo y poder tomar acciones que sobre escriban a las decisiones 
tomadas por el sistema. Se probó la efectividad de la lógica difusa para controlar sistemas no lineales, además de optimizar el uso de recursos en un 
invernadero \cite{low_cost_fuzzy_logic_greenhouse}.

% agregar mas antecedentes

% buscar antecedentes de sistemas existentes que sean costosos para que la metrica sea el bajo costo del objetivo

\subsection{Definición del problema}

% si el objetivo es disminuir el costo aqui puedo argumentar que el costo de sistemas es muy alto
% restringir las afirmaciones que se hagan a donde se implementara el invernadero. Usar encuestas del proyecto de PITI para demostrar que las personas de

%% Sistema de control y monitoreo de un invernadero basado en el Internet de las cosas y logica difusa



% problema de investigacion, no del contexto. No esta ligado directamente al objetivo
La agricultura en Samalayuca, Chihuahua, se basa, en su mayoría, en métodos tradicionales o populares. Esto dificulta que los agricultores
locales puedan competir contra extranjeros que producen en ambientes controlados y automatizados, ya que no pueden igualar ni la calidad 
del producto ni la cantidad producida. Lamentablemente, esto afecta aún más a pequeños y medianos productores ya que los sistemas de control 
y monitoreo son muy costosos.

% La agricultura en Samalayuca, Chihuahua, se basa, en su mayoría, en métodos tradicionales o populares. Esto impacta de forma negativa a los 
% productores de pequeña o mediana escala ya que no cuentan con las herramientas para tomar decisiones basadas en información. Al no contar con 
% información, el replicar resultados en temporada tras temporada es complicado.

% Una toma de decisiones informada ayudará a los agricultores a utilizar de una forma más eficiente sus recursos, disminuyendo el costo de producción. 
% Además, al llevar sistematizar la producción se tendrán resultados replicables temporada tras temporada.

\subsection{Objetivo}
% lo que diga mi objetivo tiene que resolver mi problema al 100% - Como medir el resultado? 
% Utilizando tecnicas de logica difusa
Implementar un sistema de bajo costo para controlar y monitorear el clima de un invernadero utilizando técnicas de lógica difusa.