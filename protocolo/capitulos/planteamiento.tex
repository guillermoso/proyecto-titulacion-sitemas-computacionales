\section{Planteamiento del problema}

\subsection{Antecedentes}

% red inalambrica de bajo costo para el monitoreo de una plantación de olivos
En \cite{olive_orchard_monitorization} se implementó una red inalámbrica de sensores de bajo costo para monitorear la temperatura, humedad y concentración de gases del ambiente, además de la humedad del suelo en una plantación de olivos. Se utilizó el sensor DHT 11 para monitorear la temperatura y humedad del ambiente, un higrómetro para medir la humedad del suelo y un MQ-135 para detectar gases nocivos (CO2, amoniaco, benzeno, alcohol y humo del fuego). Cada nodo de esta red cuenta con los sensores para realizar las lecturas mencionadas, un microcontrolador (ESP8266) y una antena para habilitar la comunicación mediante el protocolo LoRaWAN, todo alimentado por celdas fotovoltaicas. Para el monitoreo de las lecturas, se desarrollaron una aplicación móvil y una interfaz web que muestran los datos en tiempo real.

Es importante monitorear de forma separada la humedad del ambiente como la humedad del suelo debido a que estas no están correlacionadas. Ambas se relacionan a enfermedades y plagas, sin embargo, la humedad del ambiente influye en la aparición de enfermedades y la humedad del suelo en la aparición de plagas. También es muy valioso contar con mediciones de temperatura cerca del cultivo. De esta manera se tienen datos más precisos que los que proveen los sitios/aplicaciones meteorológicas \cite{olive_orchard_monitorization}.

% red inalambrica de bajo costo para monitorear gases de invernadero
Asimismo, en \cite{wsn_greenhouse_gases}, se desarrolló una red inalámbrica de sensores. Sin embargo, esta vez se utilizó para monitorear la temperatura y la concentración de gases dentro de un invernadero. La red consistió de dos nodos y una computadora central, comunicándose a través del protocolo ZigBee. Los nodos de la red consisten de un sensor de temperatura, sensores de monóxido de carbono, dióxido de carbono y metano para el monitoreo de gases, un transmisor ZigBee y un GPS (integrado en el transmisor). Ambos nodos fueron capaces de realizar lecturas de los sensores gracias un microcontrolador (Arduino uno). La computadora central consistió del mismo microcontrolador y un recibidor del protocolo ZigBee para recolectar los datos de los nodos. Estos datos se guardan en una base de datos de MySQL. El monitoreo de los datos se realiza en tiempo real a través de una página web.

% sistema de bajo costo con logica difusa
En la Universidad de Magdalena, Colombia se creó un sistema de bajo costo para monitoreo y control remoto de un invernadero utilizando lógica difusa \cite{low_cost_fuzzy_logic_greenhouse}. La utilización de lógica difusa en sistemas de control permite traducir variables a conjuntos previamente definidos que contienen la terminología difusa como muy frío, frío, caliente, muy caliente, etcétera. Gracias a esta traducción se pueden tomar acciones más precisas para realizar el control \cite{agriculture_automation_review}. El algoritmo difuso se implementó en un microcontrolador (Arduino Mega) para tomar acciones de control sobre la temperatura y humedad del ambiente, humedad del suelo e iluminación. Además, se diseño un sitio web para realizar el monitoreo y poder tomar acciones que sobre escriban a las decisiones tomadas por el sistema. Se probó la efectividad de la lógica difusa para controlar sistemas no lineales, además de optimizar el uso de recursos en un 
invernadero \cite{low_cost_fuzzy_logic_greenhouse}.

% sistema de control para un cultivo hidroponico con logica difusa
En \cite{fuzzy_logic_controller} se desarrolló un sistema para controlar, mediante técnicas de lógica difusa, la cantidad, el pH y la conductividad eléctrica (EC) de la solución nutritiva y la temperatura del ambiente de un cultivo hidropónico. Todas las variables mencionadas anteriormente se tomaron en cuenta para definir los conjuntos difusos. Después de determinar el grado de pertenencia de cada lectura dentro su correspondiente conjunto, se toman acciones de corta, mediana y larga duración, así como no tomar acción alguna si es que las lecturas están dentro de los rangos aceptables. La duración de la acción determina el tiempo que se tomará para ajustar el parámetro. Por ejemplo, si el pH de la solución es muy ácido y el nivel del agua es muy alto, se determina que se tomará una acción de larga duración para aumentar el pH de la solución ya que se necesita una mayor cantidad de líquido ajustador para afectar el pH de un mayor volumen de solución. 

Este sistema de control \cite{fuzzy_logic_controller} se basó en sensores y microcontroladores de bajo costo. Se utilizó un Arduino uno para realizar las mediciones, el control difuso y el envío de datos a un ESP8266 para que este envíe los datos a través de internet a un servidor en la nube.

En una investigación conjunta entre la Universidad Sains y Universidad de tecnología MARA de Malasia \cite{fuzzy_systems_agriculture}, se desarrolló un prototipo para automatizar el riego de un cultivo basado en lógica difusa. El monitoreo de parámetros se realizó a través de una red inalámbrica de sensores. Esta red se conformó por 3 nodos, un controlador maestro, un nodo para monitorear la cantidad de agua disponible en un depósito y un controlador para activar o desactivar la bomba de agua que activa el riego. El controlador maestro es el que se encarga de realizar el control difuso. Las entradas de las reglas difusas son las siguientes: humedad relativa, temperatura del ambiente y humedad del suelo. Después de la evaluación de las reglas, se obtiene el tiempo de riego (corto, mediano o largo) dentro del rango de 0 a 15 segundos. En caso de que no haya agua en el depósito, no se activará el riego.

% TODO: agregar el último antecedente  \ref{fuzzy_greenhouse_experimentaly_validated}

\subsection{Definición del problema}
El costo de los sistemas que se encuentran en el mercado supera en 300\% lo que los agricultores encuestados de Samalayuca, Chihuahua, están dispuestos a pagar. El no contar con un sistema de control y monitoreo dificulta que los agricultores locales puedan competir contra extranjeros que producen en ambientes controlados y automatizados, ya que no pueden igualar ni la calidad del producto ni la cantidad producida.