\section*{Introducción}

"La agricultura es la base de la economía en muchos países, esta provee a la humanidad de algunas de sus necesidades más básicas: comida y fibra" \cite{appsremotesensing}. La demanda de comida y productos derivados de la agricultura está proyectada a incrementar en un 70\% para el año 2050 \cite{wik_pingali_brocai_2008}. La agricultura de precisión (PA por sus siglas en inglés) es clave para poder alcanzar la agricultura sustentable. La PA se puede definir como una estrategia de manejo que implementa la recopilación de información, comunicación entre sistemas y técnicas de análisis de datos para apoyar la toma de decisiones, por ejemplo: aplicación de fertilizantes, control de plagas, control de riego, identificación de enfermedades, entre otros \cite{appsremotesensing}.

Actualmente existen varios sistemas en el mercado para el control y monitoreo de invernaderos o cuartos de producción. Los sistemas básicos ofrecen monitoreo de temperatura, humedad, CO2, e intensidad de la luz, con un costo de \$2,159.90 USD y \$2,351.00 USD respectivamente \cite{intelliclimate_kit_2021}, \cite{smartbee_kit_2021}. Ambos sistemas se pueden conectar a internet para realizar un monitoreo remoto, alertar cuando las condiciones salgan de los límites definidos e iniciar las acciones correctivas correspondientes, sin embargo, \cite{intelliclimate_kit_2021} necesitará una interfaz y una subscripción extra, añadiendo \$279.00 USD al costo. 

Existen otras alternativas, sin embargo, estas ofrecen un servicio similar excepto a escala industrial \cite{ceres_greenhouse_solutions_2021}, \cite{autogrow_climate_control_2021}, \cite{climate_control_2021}. Debido a su mercado objetivo, estos sistemas industriales solo ofrecen cotizaciones a clientes potenciales.

En junio de 2020 se realizó una encuesta (\ref{encuesta}) en la que participó un grupo de agricultores de escala pequeña y mediana, ubicados en Samalayuca, Chihuahua. El 87.6\% de los sondeados no cuentan con un sistema de riego automatizado. Al 76.1\% les gustaría un sistema de ventilación automatizado. Los primeros tres parámetros que se consideraron más importantes de monitorear fueron los siguientes: temperatura del ambiente, humedad del ambiente y la humedad del suelo. Al 84.1\% les gustaría controlar y monitorear sus cultivos desde su teléfono o computadora, sin embargo, casi el 90\% de los encuestados no están dispuestos a adquirir un sistema a costo del mercado.
