\begin{abstract} 
    \noindent
    En la actualidad existen asistentes de voz que son capaces de reconocer comandos o instrucciones a partir de la voz de las personas. Dichos asistentes son usados principalmente para tareas sencillas como búsquedas en Internet o agendas y aún son ineficientes a la hora de reconocer y procesar el contexto en una conversación. En este proyecto se propone implementar un asistente de voz con la capacidad de interpretar el lenguaje natural y seguir una conversación sin salirse de contexto, utilizando \textit{answer sets programming} (ASP). En este documento se muestra la descripción de los asistentes de voz más relevantes actualmente, se da una justificación del proyecto, la metodología, los objetivos generales y el coronograma de actividades a seguir para implementar un prototipo de asistente de voz.
\end{abstract}
     