\section{Justificación}


Este proyecto a los agricultores de Samalayuca al proporcionarles herramientas de bajo costo para tomar decisiones basadas en información. 
Gracias a ellas podrán mejorar la calidad y cantidad de sus productos permitiéndoles satisfacer la creciente demanda de alimentos.

La automatización dentro del invernadero contribuirá a su mejor manejo, disminuyendo así los costos de mano de obra.



Además, el control y 
monitoreo remoto les facilitará el trabajo al poder controlar desde c


 les permitirá a los productores de Samalayuca disponer una alternativa de bajo costo para controlar y monitorear 

Este proyecto a los productores de Samalayuca al disponer de una alternativa de bajo costo.

%Un parrafo debe llevar almenos 2 oraciones
[Expresa las razones por las cuáles crees/supones es necesario desarrollar el proyecto plasmado en esta propuesta. 
Deberás describir y dimensionar la necesidad, problema u oportunidad en la cual se centra. 
Se tiene que establecer la importancia y grado de generalización que puede hacerse de los resultados del proyecto, así como los beneficios derivados de su realización. 
Aquí deberá incluirse el porqué vale la pena atender el problema, en qué pueden ayudar los resultados, y quiénes se beneficiarán de los mismos. 
También hay que incluir aquí el impacto que pueda tener en diferentes ámbitos, como el  tecnológico, académico, tecnológico, ecológico, económico, social, ambiental, y así, según corresponda. 
Se presentan los beneficios que se obtendrían al lograr el objetivo del proyecto de investigación, demostrando por qué es necesario y significativo. 
Ayuda a identificar las razones que motivan a la realización de este proyecto.]
\begin{description}
\item [Impacto Social:] [explica
\dots]


\item [Impacto Tecnológico:]
[explica
\dots]

\item [Impacto Académico:] 
[explica \dots]

\vdots


\end{description}