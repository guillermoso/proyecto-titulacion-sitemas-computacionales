\section{Marco referencial}
%\caption[Texto en lista de figuras sin cita]{Texto en título de imagen con cita.}

% Presentar los conceptos necesarios para entender la investigacion

% principalmente de la teoria (logica difusa, invernaderos)

% en el marco teorico se describe de manera general los temas. por ejemplo, poner porque es importante medir la humedad en el invernadero y describir como funcionan algunos sensores
% de humedad, como se mide, como se manda la señal (analoga o digital) etc

% marco tecnologico - se puede describir cuales sensores o tecnologias se utilizaran

% [Presentar la teoría que va a fundamentar el proyecto con base en le planteamiento del problema que se ha realizado, los conceptos y conocimientos necesarios para desarrollar el proyecto. Se puede dividir en sub-secciones por cada concepto que se necesite, pero no olvides introducir antes con un párrafo.]

% buscar libros y tratar de parafrasear y definir con nuestras palabras los conceptos.
% utilizar imagenes. toda imagen que aparezca tiene que estar descrita en el texto. Uitilizar etiquetas
% y referencias

% dejar las referencias para el final. En los antecedentes es importante el trabajo, por lo que se mencionan al principio.
% aqui lo que importa es la informacion, por lo que s eponen al final.




\subsection{Marco teórico}

\subsubsection{Punto de acceso}
\subsubsection{DHCP}
\subsubsection{DNS}
\subsubsection{HTTP}
\subsubsection{HTTPS}
\subsubsection{JSON}
\subsubsection{API REST}
\subsubsection{Microcontroladores}
\subsubsection{\textit{Proxy}}
\subsubsection{\textit{Proxy} inverso}
\subsubsection{Servidor privado virtual}


\subsubsection{Metodología ágil} % como se explica que es aqui ya no hay que explicar lo que es en la metodologia 

\subsection{Marco tecnológico}

\subsubsection{\textit{Raspberry pi}}
Un \textit{raspberry pi} es una computadora de bajo costo del tamaño de una tarjeta de crédito. Fue creada por la fundación Raspberry Pi, con el objetivo de fomentar la enseñanza de la computación y la programación \ref{what_is_raspberry}.

Cuenta con una arquitectura móvil (ARM), por lo que pueden utilizar varias distribuciones de Linux como sistema operativo. Las principales distribuciones que están disponibles son Raspberry Pi OS, Debian Buster y Ubuntu 20.04 \ref{raspberry_os}.

\subsubsection{\textit{Hologram Nova}}
\subsubsection{Node MCU-ESP826}
\subsubsection{\textit{FastAPI}}
\subsubsection{NGINX}
\subsubsection{PM2}
\subsubsection{AJAX}
\subsubsection{MongoDB}
Motor de base de datos de código abierto no relacional que almacena los datos en estructuras tipo JSON llamados documentos. Los campos de los documentos pueden variar de documento a documento. Esta flexibilidad que ofrece contribuye a un desarrollo más rápido al no tenerse que preocupar de la integridad referencial. Tiene una licencia que permite su uso en aplicaciones personales y comerciales, siempre y cuando no se ofrezca la base de datos como un servicio \cite{what_is_mongo}.

\subsubsection{\textit{JSON Web Token}}


\subsubsection{Vue.js}

% microcontroladores

% ssh



% TODO: El marco teorico es mas para 


% 1. Agricultura de precision
%     1.1 Intro (hablar general de la agricultura, invernaderos y agricultura protegida)
%     3.1 Importancia
%     3.2 Ventajas
%     3.2 Desventajas % costo
% 2. Hardware 
%     IoT, arduinos sensores
% 3. Control
%     Intro (teoria del control, control automatico, control difuso)
%     3.1 Logica difusa
% 4. Aplicaciones web
%     4.2 Bases de datos
%     4.3 APIs

