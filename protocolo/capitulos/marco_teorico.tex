\section{Marco referencial}
%\caption[Texto en lista de figuras sin cita]{Texto en título de imagen con cita.}

% Presentar los conceptos necesarios para entender la investigacion

% principalmente de la teoria (logica difusa, invernaderos)

% en el marco teorico se describe de manera general los temas. por ejemplo, poner porque es importante medir la humedad en el invernadero y describir como funcionan algunos sensores
% de humedad, como se mide, como se manda la señal (analoga o digital) etc

% marco tecnologico - se puede describir cuales sensores o tecnologias se utilizaran

% [Presentar la teoría que va a fundamentar el proyecto con base en le planteamiento del problema que se ha realizado, los conceptos y conocimientos necesarios para desarrollar el proyecto. Se puede dividir en sub-secciones por cada concepto que se necesite, pero no olvides introducir antes con un párrafo.]

% buscar libros y tratar de parafrasear y definir con nuestras palabras los conceptos.
% utilizar imagenes. toda imagen que aparezca tiene que estar descrita en el texto. Uitilizar etiquetas
% y referencias

% dejar las referencias para el final. En los antecedentes es importante el trabajo, por lo que se mencionan al principio.
% aqui lo que importa es la informacion, por lo que s eponen al final.




\subsection{Marco teórico}


\subsection{Marco tecnológico}


% sensores
\subsubsection{}

% microcontroladores

% ssh



% TODO: El marco teorico es mas para 


1. Agricultura de precision
    1.1 Intro (hablar general de la agricultura, invernaderos y agricultura protegida)
    3.1 Importancia
    3.2 Ventajas
    3.2 Desventajas % costo
2. Hardware 
    IoT, arduinos sensores
3. Control
    Intro (teoria del control, control automatico, control difuso)
    3.1 Logica difusa
4. Aplicaciones web
    4.2 Bases de datos
    4.3 APIs

